% !tex root = ../Thesis.tex

\color{red}
To Do:
\begin{itemize}
	\item Describe what pitch correction is
	\item Describe why it is difficult
	\begin{itemize}
		\item Pitch detection is not trivial
		\item Why you can't just speed up sound and slow it down
		\item Convey the essence of the problem
	\end{itemize}
\end{itemize}
\color{black}

\section{Motivation}

\color{red}
To do:
\begin{itemize}
	\item One small out of tune note causes a whole new re-recording
	\item Quick fix after recording
	\item Live performers can now sing
	\item Produces a quirky robotic effect that can be desirable
	\item Case against audio pitch correction
	\begin{itemize}
		\item Less freedom to professional singers
		\item Vibrato, colour etc
		\item Conceals talent
	\end{itemize}
	\item Biggest case for: Interesting mathematical project
\end{itemize}
\color{black}

\section{History of Audio Pitch Correction}

\color{red}
To Do:
\begin{itemize}
	\item Eventide Harmonizer in the 70s (Only post correction)
	\item Andy Hildebrand invented AutoTune in 1997 while working on seismic data
	\item Open source pitch correction AutoTalent by Tom Baran
\end{itemize}
\color{black}
