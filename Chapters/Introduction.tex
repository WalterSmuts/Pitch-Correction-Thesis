% !tex root = ../Thesis.tex

\section{Problem Statement}

Musicians are faced with the challenge of creating good music. This comes down to
controlling many different qualities of the sound they produce. The tempo, rhythm,
pitch and volume are some examples of these qualities. They often make mistakes
under the pressure of live performances or even when making studio recordings.
Pitch is specifically hard to control for vocalists or musicians who play
instruments in the horn or fretless string families. Errors in pitch accuracy are
known as intonation errors and effect both the harmonic and melodic structure of
the music, degrading it's aesthetic appeal. Since all the efforts of the musician
culminates in a sound wave, it is subject to analysis and manipulation by the
tools, algorithms and effects developed in the field of signal processing.

Pitch correction is an effect applied to audio that attempts to fix errors in
intonation. This involves minimally changing the waveform to have the desired
pitch while retaining all the other qualities of the waveform. This effect is
mostly discussed in the context of music, where pitch is considered an important
quality, but some considerations in other fields such as speech comprehension can
be conceived.

\begin{figure}[b!]
	\includegraphics[width=\textwidth]{MusicNotation}
	\includegraphics[width=\textwidth,trim={3.5cm 0cm 2.8cm 0cm}]
	{IntonationError}
	\caption{Illustration of Intonation Errors in a Pitch Contour Diagram}
	\label{fig:IntonationError}
\end{figure}

Figure \ref{fig:IntonationError} shows an example of what is meant by errors in
intonation. The graph is a pitch contour diagram, plotting the quality of pitch
over time. One of the goals of the musician is to make the pitch contour follow
the pitch indicated by music notation as closely as possible during the duration
of the note. The red regions indicate errors in intonation, i.e.  the times which
the pitch contour does not follow the pitch indicated by the note. The goal of
pitch correction is to alter the waveform to minimize these regions of intonation
error while minimally affecting other properties of the waveform.

\color{red}
To Do:
\begin{itemize}
	\item Describe why it is difficult
	\begin{itemize}
		\item Pitch detection is not trivial
		\item Why you can't just speed up sound and slow it down
		\item Convey the essence of the problem
	\end{itemize}
		\item Motivate why it's needed
		\begin{itemize}
		\item One small out of tune note causes a whole new re-recording
		\item Quick fix after recording
		\item Live performers can now sing
		\item Produces a quirky robotic effect that can be desirable
		\item Case against audio pitch correction
		\begin{itemize}
			\item Less freedom to professional singers
			\item Vibrato, colour etc
			\item Conceals talent
		\end{itemize}
		\item Biggest case for: Interesting mathematical project
	\end{itemize}
\end{itemize}
\color{black}

\section{History of Audio Pitch Correction}

One of the first occurrences of pitch manipulation in music, at least the first
occurrence that was found by the author, was of a \footnote{The Chipmunk Song:
\url{https://www.youtube.com/watch?v=b3p7tZw6Mps}}song called ``The Chipmunk
Song'' from the animated music group ``Alvin and the Chipmunks''. The goal was to
raise the pitch of the voice of a singer to sound an octave higher than his actual
voice. This was accomplished by recording the song at half the wanted tempo and
playing back the recording at twice the speed when mixing. The technology used in
audio recordings was still analog tape recordings and the whole process was done
using these tapes. The effect of speeding up a recording to achieve a pitch shift
became known as the ``Chipmunk Effect''. The approach earned the group two
\footnote{\url{https://www.grammy.com/grammys/artists/ross-bagdasarian-sr}}Grammy
awards in 1958.

In 1977 Eventide introduced a new product, the
\footnote{\url{https://www.eventideaudio.com/products/clockworks-legacy/harmonizer/h949-harmonizer}}Eventide
H949 Harmonizer, capable of incremental pitch shifting. This was sold as a
``de-glitch pitch shifter'' and was intended to be used to fix intonation errors
and add harmonization effects after recording. Other features commonly used was to
stretch the time of radio recordings of advertisements to be an exact duration
without affecting the pitch.  The pitch shifting was implemented using single
side-band modulation techniques and was done digitally.

\begin{figure}[h]
	\centering
	\includegraphics[width=0.9\textwidth,trim={0mm 55mm 0mm 55mm},clip]
	{EventideH949}
	\caption{Eventide H949 Harmonizer Rack Mounted Unit}
	\label{fig:EventideH949}
\end{figure}

In 1996 Andy Hildebrand, an electrical engineer, was investigating seismic data
when he realised that the same techniques he was using to investigate the data
could be used to alter the pitch of audio files. His techniques for detecting
pitch, using a simplification of the autocorrelation function, was considered
\footnote{\url{https://priceonomics.com/the-inventor-of-auto-tune}}superior to the
state of the art at that time. He implemented the first version of his pitch
correcting algorithm on his Macintosh computer. His first demonstration was
considered a success and the company ``Antares Audio'' was founded. A patent was
filed in 1997 for a ``Pitch detection and intonation correction apparatus and
method''. They created a product called ``AutoTune'' which was popularised in 1998
by Cher in her song ``Believe'' which made heavy use of the AutoTune pitch
correction effect. AutoTune was the first product capable of automatic real time
pitch correction\cite{AutoTunePatent}.

\begin{figure}[h]
	\centering
	\includegraphics[width=0.9\textwidth,trim={30mm 20mm 30mm 30mm},clip]
	{AutoTuneRack}
	\caption{AutoTune Rack Mounted Unit}
	\label{fig:AutoTuneRack}
\end{figure}

AutoTune comes in a rack mounted unit for live performances shown in figure
\ref{fig:AutoTuneRack} or as a VST plugin. The software has evolved to allow for
many more features than the original pitch correction effect that the name
suggests. Modern AutoTune is still considered the state of the art product by
audio engineers.

In 2009 Tom Baran, an electrical engineer, wrote an open source pitch corrector.
This pitch corrector was written in the C programming language as a VST (Virtual
Studio Technology) plugin. At least two ports have been written and are up on
GitHub. The fact that this project is open source allows for inspection of the
code and provides insight in what design choices has been made. It uses an
autocorrelation approach to find the pitch, a PSOLA algorithm to shift the pitch
and a cubic spline interpolation algorithm for re-sampling\cite{AutoTalent}.

Finally Smule, a mobile phone application, was released in 2013. It's an
application for amateur singers to record themselves singing songs and sharing the
recordings on social media. The application features a pitch correction effect to
improve the quality of the recording before it is shared. This pitch correction is
done based on knowing what the harmony of the current note should be since it uses
a pre-recorded backing track. A patent was filled by Smule in 2011 for
``Pitch-correction of vocal performance in accord with score-coded
harmonies''\cite{SmulePatent}.

\section{Approach Taken}

\color{red}
To Do:
\begin{itemize}
	\item Discuss structure of report
	\item Show final implementation
	\item Summarise results and conclusion section
\end{itemize}
\color{black}
