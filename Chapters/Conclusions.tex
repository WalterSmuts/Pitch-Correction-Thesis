% !tex root = ../Thesis.tex

\section{Findings}

The objective of this thesis was to design and test a pitch correction system.
Metrics were designed to be a proxy measurement for aesthetic improvement of the
recordings. These metrics tried to capture the improvement in pitch accuracy and
the distortion caused by the effect. A frequency detector metric was designed to
measure the robustness the frequency detection module has towards additive white
noise.

Two frequency detection modules were implemented. One implementation was based on
using the zero-crossing method and the other was based on an autocorrelation
algorithm. The zero-crossing method displayed more noise robustness than the
autocorrelation method. The zero-crossing method required a signal to noise ration
of 4.5db while the autocorrelation method required a ratio of 17.8db.

Two frequency scaling modules were implemented. One using a very basic overlap and
add approach, and another using a sophisticated phase vocoder approach. From tests
in the implementation chapter, it was obvious that the overlap and add method was
not an appropriate method to use for pitch correction. The overlap and add method
produced a pitch shifting resolution of greater than 3 semitones while the phase
vocoder approach showed pitch shifting resolution greater that the test could
measure.

Due to time constraints, the metrics developed were only tested on one combination
of sub-modules. This combination is that of the zero-crossing method as a
frequency detection module and the phase vocoder as the frequency scaling module.
This combination produced a pitch improvement factor of 4.38 and a similarity
percentage of 44\% to the uncorrected recording. Real vocal recordings were also
tested, proving the system can work with real data. Some manipulation and
re-recording was required to produce satisfactory results.

Due to the facts presented above, the pitch correction system is not yet
considered robust to real audio recordings. It works in laboratory conditions and
requires cherry picked data to perform well. This report is considered a stepping
stone towards a system that can perform well in a real world scenario. Further
investigation is required to produce a system that has the desired effect.
Recommendations are made to inform such an investigation.

\section{Recommendations}



\begin{itemize}

\item
The design of the system as a whole needs to be relative to a time base.
This means sending a time-base array, t, along with any calculation done on a
segment This will make some implementation details easy and elegant. Debugging
will also be much easier.

\item
A frame based approach needs to be taken more seriously. Currently the
system calculates the contours separately and passes information about the signal
as a whole between functions. This was easier to implement but caused much larger
hassle later on. Functions passing single frames will work much easier.

\item
The state of the art for small pitch shifting is considered the synchronised
overlap and add algorithm. The next frequency scaling algorithm to implement
should be the synchronised overlap and add approach.

\item
The autocorrelation frequency detector should outperform the zero-crossing
method. The reason it does not is most probably due to interpolation being
required to provide enough resolution to function properly.

\item
The autocorrelation frequency detector needs to be tuned scientifically. The
threshold values are currently picked from intuition but tests to calculate their
optimal value is considered to have a chance of significantly improving
performance.

\item
The inelegant mapping algorithm used by the phase vocoder, mapping an
arbitrary frequency scaling contour to sample points for the STFT array and spline
interpolation function requires some investigation. An email was sent to Professor
Dan Ellis, the creator of the phase vocoder implementation being used, to get some
ideas. He responded(too late) with a link to a GitHub
repository\footnotemark\space that has a correct implementation of the time
stretching counterpart of the mapping function.

\footnotetext{\url{https://github.com/dpwe/pitchfilter/blob/master/resample_map.m}}

\item
Some parameters, such as pitch acceptableness error, was taken at an
arbitrary value. These values need to be determined psychologically.

\end{itemize}
