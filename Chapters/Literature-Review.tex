% !tex root = ../Thesis.tex

This literature review attempts to cover all the subjects relevant to pitch
correction. Musical theory is investigated first, covering the topics of human
pitch perception and tuning. This section attempts to determine what it means for
pitch to be correct and how this relates to frequency. After it is well known what is musically considered to be a correct pitch, the
general structure of how pitch correction is approached is investigated. This
structure naturally splits up into modules and each of these modules is
investigated further.

\section{Music Theory}

The intent of this section is to come up with a definition of what will be
considered correct pitch. Some foundational work is required to define basic
musical concepts in a rigorous way. To start we need to define exactly what is
meant by a musical note.

A note is a sound made from a musical instrument. It has a pitch, volume, timbre
and length. These are the characteristics a musician would consider when composing
a piece of music. Each of these characteristics have a more rigorous scientific
counterpart they are related to. Pitch relates to frequency; Volume relates to
amplitude; timbre relates to harmonic content and length relates to duration. The
characteristic pair relevant to pitch correction is the pitch and frequency pair
and needs to be covered in more depth.

\begin{wrapfigure}{L}{0.5\textwidth}
\includegraphics[width=0.5\textwidth,trim={3mm 0mm 20mm 8mm},clip]{Frequency-Vs-Pitch}
\caption{"Frequency vs Pitch"}
\label{fig:FrequencyVsPitch}
\end{wrapfigure}

Pitch and frequency are terms often used interchangeably. Pitch is a
\textit{sensation} experienced by humans when they hear notes containing
frequencies between 31Hz and 17.6 kHz\cite{Hearing}. The sensation of pitch is
scaled on an subjective ``high'' and ``low'' scale. The higher the frequency, the
higher the pitch perceived and the lower the frequency, the lower the pitch
perceived. This relationship between pitch and frequency is generally considered
to be logarithmic. Slight deviations from the expected logarithmic relationship
was found \cite{PitchVsFrequency} but was deemed minor and unnecessary to
incorporate for this project.

In Figure \ref{fig:FrequencyVsPitch} the relationship of frequency and pitch is
shown. Pitch is generally denoted by letters ranging from A to G with
sharps(\musSharp) and flats(\musFlat) called accidentals.  This is due to a long
history of convention and is irrelevant for now. The main takeaway is that the
perceived change in pitch is constant for each successive note. From this graph it
can be seen that frequency is exponentially dependent on pitch, or inversely,
pitch is logarithmically dependent on frequency.

\color{red}
To do:
\begin{itemize}
	\item In tune
	\item Harmony
	\item Tuning system and equal tempered tuning
	\item Tie back to the fact that this gives us a definition of correct
\end{itemize}
\color{black}

\section{General Pitch Correction Structure}

\color{red}
To Do:
\begin{itemize}
	\item Stages and sections needed e.g.:
	\begin{itemize}
		\item Segmentation/Windowing
		\item Frequency detection
		\item Decide where to shift
		\item Frequency scaling
	\end{itemize}
	\item Add a flow diagram
\end{itemize}
\color{black}

\section{Segmentation}

\color{red}
To do:
\begin{itemize}
	\item Describe what segmentation is and why it is necessary
	\item Describe stages
	\begin{itemize}
		\item Split (overlap)
		\item Do computation
		\item Stitch (overlap and add)
	\end{itemize}
	\item Reason to overlap and size of overlapping (TRADE-OFF)
	\item Window size (TRADE-OFF)
	\begin{itemize}
		\item Small means low resolution
		\item Large means latency
		\item Small generally means more computation
	\end{itemize}
	\item Properties to preserve relevant to real time auto tuning
\end{itemize}
\color{black}

\section{Frequency Detection}

\color{red}
To do:
\begin{itemize}
	\item Explain that there is no single go-to method
	\item Different methods have different characteristics
	\item What is important for this application
	\item Give overview of which methods will be investigated
\end{itemize}
\color{black}

\subsection{Zero Crossing Method}

\color{red}
To do:
\begin{itemize}
	\item Describe method roughly
	\item More in depth description comes in implementation section
\end{itemize}
\color{black}

\subsection{Periodogram Method}

\color{red}
To do:
\begin{itemize}
	\item Describe method roughly
	\item More in depth description comes in implementation section
\end{itemize}
\color{black}

\subsection{Max FFT Method}

\color{red}
To do:
\begin{itemize}
	\item Describe method roughly
	\item More in depth description comes in implementation section
\end{itemize}
\color{black}

\subsection{YIN's Method}

\color{red}
To do:
\begin{itemize}
	\item Describe method roughly
	\item More in depth description comes in implementation section
\end{itemize}
\color{black}

\section{Frequency Scaling}

\color{red}
To do:
\begin{itemize}
	\item General approach is to expand/extrapolate frequency
	\item Time domain approaches and frequency domain approaches
	\item Introduce two approaches Phase Vocoder and pitch synchronous overlap and add
\end{itemize}
\color{black}

\subsection{Phase Vocoder}

\color{red}
To do:
\begin{itemize}
	\item Describe method roughly
	\item More in depth description comes in implementation section
\end{itemize}
\color{black}

\subsection{PSOLA}

\color{red}
To do:
\begin{itemize}
	\item Describe method roughly
	\item More in depth description comes in implementation section
\end{itemize}
\color{black}
