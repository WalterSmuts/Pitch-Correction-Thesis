% !tex root = ../Thesis.tex

This literature review attempts to cover all the subjects relevant to pitch
correction. Musical theory is investigated first, covering the topics of human
pitch perception and tuning. This section attempts to determine what it means for
pitch to be correct and how this relates to frequency.

After it is well known what is musically considered to be a correct pitch, the
general structure of how pitch correction is approached is investigated. This
structure naturally splits up into modules and each of these modules is
investigated further.

\section{Music Theory}

The intent of this section is to understand what it means for a pitch to be
correct and how this relates to frequency. The term ``in tune'' is often used in
the musical world and describes a note that sounds good in relation to other notes
played either before, after or during the note being considered. Finally the idea
of a tuning system is investigated which produces a list of available notes that
are correct and can be used by the pitch corrector when choosing the correct note
to shift to.

Pitch and frequency are terms often used interchangeably. Pitch is a sensation
experienced by humans when they hear sounds containing frequencies between 31Hz
and 17.6 kHz\cite{Hearing}. The sensation of pitch is scaled on an subjective
``high'' and ``low'' scale. The higher the frequency, the higher the pitch
perceived and the lower the frequency, the lower the pitch perceived. This
relationship between pitch and frequency is generally considered to be
logarithmic.  Slight deviations from the expected logarithmic relationship was
found \cite{PitchVsFrequency} but was deemed minor and unnecessary to incorporate
for this project.

\color{red}
To do:
\begin{itemize}
	\item Logarithmic relationship means distance metric.
	\item Draw graph that shows this.
	\item Explain note harmony
	\item Describe tuning and equaled tempered tuning
	\item Describe how this is relevant to pitch correction
\end{itemize}
\color{black}

\section{General Pitch Correction Structure}

\color{red}
To Do:
\begin{itemize}
	\item Stages and sections needed e.g.:
	\begin{itemize}
		\item Segmentation/Windowing
		\item Frequency detection
		\item Decide where to shift
		\item Frequency scaling
	\end{itemize}
	\item Add a flow diagram
\end{itemize}
\color{black}

\section{Segmentation}

\color{red}
To do:
\begin{itemize}
	\item Describe what segmentation is and why it is necessary
	\item Describe stages
	\begin{itemize}
		\item Split (overlap)
		\item Do computation
		\item Stitch (overlap and add)
	\end{itemize}
	\item Reason to overlap and size of overlapping (TRADE-OFF)
	\item Window size (TRADE-OFF)
	\begin{itemize}
		\item Small means low resolution
		\item Large means latency
		\item Small generally means more computation
	\end{itemize}
	\item Properties to preserve relevant to real time auto tuning
\end{itemize}
\color{black}

\section{Frequency Detection}

\color{red}
To do:
\begin{itemize}
	\item Explain that there is no single go-to method
	\item Different methods have different characteristics
	\item What is important for this application
	\item Give overview of which methods will be investigated
\end{itemize}
\color{black}

\subsection{Zero Crossing Method}

\color{red}
To do:
\begin{itemize}
	\item Describe method roughly
	\item More in depth description comes in implementation section
\end{itemize}
\color{black}

\subsection{Periodogram Method}

\color{red}
To do:
\begin{itemize}
	\item Describe method roughly
	\item More in depth description comes in implementation section
\end{itemize}
\color{black}

\subsection{Max FFT Method}

\color{red}
To do:
\begin{itemize}
	\item Describe method roughly
	\item More in depth description comes in implementation section
\end{itemize}
\color{black}

\subsection{YIN's Method}

\color{red}
To do:
\begin{itemize}
	\item Describe method roughly
	\item More in depth description comes in implementation section
\end{itemize}
\color{black}

\section{Frequency Scaling}

\color{red}
To do:
\begin{itemize}
	\item General approach is to expand/extrapolate frequency
	\item Time domain approaches and frequency domain approaches
	\item Introduce two approaches Phase Vocoder and pitch synchronous overlap and add
\end{itemize}
\color{black}

\subsection{Phase Vocoder}

\color{red}
To do:
\begin{itemize}
	\item Describe method roughly
	\item More in depth description comes in implementation section
\end{itemize}
\color{black}

\subsection{PSOLA}

\color{red}
To do:
\begin{itemize}
	\item Describe method roughly
	\item More in depth description comes in implementation section
\end{itemize}
\color{black}
