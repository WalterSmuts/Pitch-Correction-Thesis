% !tex root = ../Thesis.tex

\color{red}
Here's an example of referencing the original phase vocoder paper
\cite{OriginalPhaseVocoder} and here's a reference to the improved phase
vocoder paper \cite{ImprovedPhaseVocoder}.

To Do:
\begin{itemize}
	\item Describe the intent of the review
	\item Descrive the structure of the review
\end{itemize}
\color{black}

\section{Music Theory}

\color{red}
To do:
\begin{itemize}
	\item Describe pitch as perceived by people
	\item Describe tuning and equaled tempered tuning
	\item Describe how this is relevant to pitch correction
\end{itemize}
\color{black}

\section{General Pitch Correction Structure}

\color{red}
To Do:
\begin{itemize}
	\item Stages and sections needed e.g.:
	\begin{itemize}
		\item Segmentation/Windowing
		\item Frequency detection
		\item Decide where to shift
		\item Frequency scaling
	\end{itemize}
	\item Add a flow diagram
\end{itemize}
\color{black}

\section{Segmentation}

\color{red}
To do:
\begin{itemize}
	\item Describe what segmentation is and why it is necessary
	\item Describe stages
	\begin{itemize}
		\item Split (overlap)
		\item Do computation
		\item Stitch (overlap and add)
	\end{itemize}
	\item Reason to overlap and size of overlapping (TRADE-OFF)
	\item Window size (TRADE-OFF)
	\begin{itemize}
		\item Small means low resolution
		\item Large means latency
		\item Small generally means more computation
	\end{itemize}
	\item Properties to preserve relevant to real time auto tuning
\end{itemize}
\color{black}

\section{Frequency Detection}

\color{red}
To do:
\begin{itemize}
	\item Explain that there is no single go-to method
	\item Different methods have different characteristics
	\item What is important for this application
	\item Give overview of which methods will be investigated
\end{itemize}
\color{black}

\subsection{Zero Crossing Method}

\color{red}
To do:
\begin{itemize}
	\item Describe method roughly
	\item More in depth description comes in implementation section
\end{itemize}
\color{black}

\subsection{Periodogram Method}

\color{red}
To do:
\begin{itemize}
	\item Describe method roughly
	\item More in depth description comes in implementation section
\end{itemize}
\color{black}

\subsection{Max FFT Method}

\color{red}
To do:
\begin{itemize}
	\item Describe method roughly
	\item More in depth description comes in implementation section
\end{itemize}
\color{black}

\subsection{YIN's Method}

\color{red}
To do:
\begin{itemize}
	\item Describe method roughly
	\item More in depth description comes in implementation section
\end{itemize}
\color{black}

\section{Frequency Scaling}

\color{red}
To do:
\begin{itemize}
	\item General approach is to expand/extrapolate frequency
	\item Time domain approaches and frequency domain approaches
	\item Introduce two approaches Phase Vocoder and pitch synchronous overlap and add
\end{itemize}
\color{black}

\subsection{Phase Vocoder}

\color{red}
To do:
\begin{itemize}
	\item Describe method roughly
	\item More in depth description comes in implementation section
\end{itemize}
\color{black}

\subsection{PSOLA}

\color{red}
To do:
\begin{itemize}
	\item Describe method roughly
	\item More in depth description comes in implementation section
\end{itemize}
\color{black}
