% !tex root = ../Thesis.tex

\color{red}
Here's an example of referencing the original phase vocoder paper
\cite{OriginalPhaseVocoder} and here's a reference to the improved phase
vocoder paper \cite{ImprovedPhaseVocoder}.

To Do:
\begin{itemize}
	\item Describe the intent of the review
\end{itemize}
\color{black}

\section{General Autotuning Structure}

\color{red}
To Do:
\begin{itemize}
	\item Stages and sections needed e.g.:
	\begin{itemize}
		\item Pitch detection
		\item Pitch shifting
		\item Windowing (Where does this come in?)
	\end{itemize}
\end{itemize}
\color{black}

\section{Music Theory}

\color{red}
To do:
\begin{itemize}
	\item Describe pitch as perceived by people and how it's relevant to auto tuning.
	\item Describe Tuning and equaled tempered tuning
\end{itemize}
\color{black}

\section{Windowing}

\color{red}
To do:
\begin{itemize}
	\item Describe what windowing is
	\item Describe stages (Split into windows, do computation and stitch)
	\item Resolution of tuning (only relevant to the size of windows)
	\item Implications of math (What is lost as window gets smaller???)
	\item Why shouldn't windows be too large
	\item Properties to preserve relevant to real time auto tuning
	\item Call it the right thing!!! Maybe segmentation
\end{itemize}
\color{black}

\subsection{Splitting Windows}

\color{red}
To do:
\begin{itemize}
	\item Is there anything to describe here?
\end{itemize}
\color{black}

\subsection{Stitching Windows}

\color{red}
To do:
\begin{itemize}
	\item After the computation is done it's not necessarily  continuous anymore
	\item How to stitch the windows back together
\end{itemize}
\color{black}

\section{Pitch Detection}

\color{red}
To do:
\begin{itemize}
	\item No single method
	\item Different methods have different characteristics
	\item What is important for this application
	\item Overview which methods will be investigated
\end{itemize}
\color{black}

\subsection{Zero Crossing Method}

\color{red}
To do:
\begin{itemize}
	\item Describe method.
\end{itemize}
\color{black}

\subsection{Periodogram Method}

\color{red}
To do:
\begin{itemize}
	\item Describe method.
\end{itemize}
\color{black}

\subsection{YIN's Method}

\color{red}
To do:
\begin{itemize}
	\item Describe method.
\end{itemize}
\color{black}

\section{Pitch Shifting}

\color{red}
To do:
\begin{itemize}
	\item Describe method.
\end{itemize}
\color{black}

\section{Links}

\begin{itemize}

\item \url
{http://chenzhe142.github.io/nu-eecs352}

\item \url
{http://ws2.binghamton.edu/zahorian/yaapt.htm}

\item\url
{https://en.wikipedia.org/wiki/Pitch\_detection\_algorithm}

\item\url
{https://docs.google.com/document/d/1tp\_YYqvUUViPUb\_ZTzGNi4BVfWNqkrOlXcdM63DAwWY/edit}

\item\url
{https://sound.eti.pg.gda.pl/student/eim/synteza/leszczyna/index_ang.htm}

\end{itemize}
