% !tex root = ../Thesis.tex

To prove that no perfect tuning system exists, is equivalent to proving the
following proposition:

\begin{equation}\label{eq:GeneralProp}
	\nexists r \in \mathbb{R} \text{ s.t. }
	R \subseteq \{ r^n | n \in \mathbb{N}_0 \}
	\text{ where } R = \{ \frac{1}{1}, \frac{15}{16}, \frac{9}{8}, \frac{6}{5}, \dots \}
\end{equation}

R is the set of all the harmonic ratios required in the wanted tuning system. Only
two of these ratios, the most fundamental ratios in any tuning system, are
required for the proof. The octave interval, ratio 2/1, and the perfect fifth, ratio
3/2. The proof approach is a proof by contradiction.

Assume there exists:
\begin{equation}\label{eq:SpecificProp}
	r^n = \frac{2}{1} \text{ and } r^m = \frac{3}{2}
	\text{ s.t. }
	r \in \mathbb{R} \text{ and } n,m \in \mathbb{N}_0
\end{equation}

Taking the $\log_{2}$ of each equation in \ref{eq:SpecificProp} the follwing two
equations are obtained:

\begin{equation}\label{eq:1}
	\log_{2}(r) = \frac{\log_{2}(2)}{n}
\end{equation}
\begin{equation}\label{eq:2}
	\log_{2}(r) = \frac{\log_{2}(3) - \log_{2}(2)}{m}
\end{equation}

Equating equation \ref{eq:1} and equation \ref{eq:2} and simplifying, the following
equation is obtained:
\begin{equation}\label{eq:2}
	\log_{2}(3) = \frac{m+n}{n}
\end{equation}

Since both n and m are elements of $\mathbb{N}_0$, the RHS is rational.
The number $\log_{2}(3)$ is known to be irration. This is a contradiction. Q.E.D.
